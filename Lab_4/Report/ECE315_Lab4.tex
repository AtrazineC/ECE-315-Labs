\documentclass[11pt, letterpaper, titlepage]{article}
\usepackage[utf8]{inputenc}
\usepackage{geometry}
\usepackage{color,graphicx,overpic} 
\usepackage{fancyhdr}
\usepackage{amsmath,amsthm,amsfonts,amssymb}
\usepackage{mathtools}
\usepackage{hyperref}
\usepackage{multicol}
\usepackage{array}
\usepackage{float}
\usepackage{blindtext}
\usepackage{longtable}
\usepackage{scrextend}
\usepackage[font=small,labelfont=bf]{caption}
\usepackage[framemethod=tikz]{mdframed}
\usepackage{calc}
\usepackage{titlesec}
\usepackage{listings}
\usepackage[normalem]{ulem}
\usepackage{tabularx}
\usepackage{mathrsfs}
\usepackage{bookmark}
\usepackage{setspace}
\usepackage{tabularx}
\usepackage{ltablex}
\usepackage{enumitem}
\usepackage{minted}
\usepackage[simplified]{pgf-umlcd}
\usepackage{apple_emoji}

\mathtoolsset{showonlyrefs}  
\allowdisplaybreaks
\definecolor{mycolor}{rgb}{0, 0, 0}

\geometry{top=2.54cm, left=2.54cm, right=2.54cm, bottom=2.54cm}
\setlength{\headheight}{20pt}
\setlength{\parskip}{0.5cm}
\setlength{\parindent}{0pt}


\definecolor{bg}{rgb}{0.95,0.95,0.95}

\usemintedstyle{vs}
\setminted{linenos, fontsize=\footnotesize}
\setmintedinline{bgcolor=bg, style=bw, fontsize=\normalsize}

% Line spacing
\renewcommand{\baselinestretch}{1.3} 

% Title page
\title{\textbf{\Huge{ 
\begin{center}
ECE 315 Lab 4 🧀
\end{center} 
}}}

\author{For Ahmed and Shyama 🎁💯🙏 \\ \\ 🚙 Lora Ma \\ 🌎 Benjamin Kong \\ \\ECE 315 Lab Section H41}

% Header/Footer
\pagestyle{fancy}
\fancyhf{}
\rhead{\thepage}
\lhead{\textit{ECE 315 - Lab 4}}
\rfoot{}

% Hyperlink colors
\hypersetup{
    colorlinks=true,
    linkcolor=blue,
    filecolor=blue,      
    urlcolor=blue,
}

\begin{document}
\maketitle
\thispagestyle{empty}
\tableofcontents 
\newpage
\pagenumbering{arabic}

\section{Abstract}
The purpose of this lab was to 
\begin{itemize}
  \item gain experience using the serial peripheral interface in both the master and slave modes
  \item gain experience creating artificial load on a CPU and then measuring the resulting load using \mintinline{text}{vTaskGetRunTimeStats()}.
\end{itemize}
We will be using the Zybo Z7 development board by Digilent. The board is built around the Xilinx Zynq-7010 System-on-Chip silicon chip and contains two 667 MHz ARM Cortex A9 32-bit CPUs. For this lab, we will be using CPU0 to run the FreeRTOS real-time kernel which will have three non-idle tasks for the SPI interface. 

In this lab, we will be completing two exercises. Exercise 1 is to verify the given system and add the byte count message. The program will echo characters that are type on the keyboard until the termination sequence \mintinline{text}{\r#\r} is detected. Then, a string will be generated and sent over SPI which will look something like \mintinline{text}{The number of characters received over SPI: <number> \n} where \mintinline{text}{<number>} is the total bytes that were received over the SPI1 interface.

In exercise 2, we will be calibrating the load generator experimentally. The program prints the percentage values of time consumed by the tasks \mintinline{text}{TaskCPULoadGen}, \mintinline{text}{TaskLoopCountProcessor}, and \mintinline{text}{TaskPrintRunTimeStats}. It will be displayed in a tabular format on the terminal. We will the utilize this program to experimentally determine the value of \mintinline{text}{loop_count}. 

\section{Design}

\subsection{Hardware Design}
We first got the hardware set up. We first set up the wire connections from the Zybo to our breadboard according to the schematics in the lab handout. We connected the appropriate wires and resistors to the slots of the chip. We then connected the motor/motor control to the breadboard as well. It is important to note that the motor is powered by an external 5V connection, not by the Zybo. We verified that 5V was running through the system using a voltmeter.

\subsection{Exercise 1 Design}
We first completed the \mintinline{text}{_Task_Motor} function in \mintinline{text}{lab4_main.c}. Inside of the \mintinline{text}{while} loop, we wait until we read some motor parameters from FIFO1.

\subsection{Exercise 2 Design}


\section{Testing}

\subsection{Exercise 1 Tests}

\subsection{Exercise 2 Tests}

\section{Conclusion}
The purpose of this lab was to 
\begin{itemize}
  \item gain experience using the serial peripheral interface in both the master and slave modes
  \item gain experience creating artificial load on a CPU and then measuring the resulting load using \mintinline{text}{vTaskGetRunTimeStats()}.
\end{itemize}

We believe we have fully completed the objectives of this lab.

In exercise 1, we gained experience using the SPI interface. The program echos characters that are typed on the keyboard until the termination sequence is detected. Then, it will generate a string to be sent over SPI where it will print the number of total bytes that were received over the SPI interface.

In exercise 2, we created a program that prints the percentage values of time consumed by three tasks. We summarized this information to be printed in a tabular format in the terminal. This program was then used to experimentally determine the value of \mintinline{text}{loop_count}.

We were able to gain experience using the serial peripheral interface (SPI) in both master and slave modes. We were also able to gain experience creating artificial load on a CPU and then measuring the resulting load using \mintinline{text}{vTaskGetRunTimeStats()}.

\section{Appendix}

\subsection{Exercise 1 Source Code}
The source code is in the file \mintinline{text}{lab4_main.c} that was submitted along with this report.

\subsection{Exercise 2 Source Code}
The source code is in the file \mintinline{text}{lab4_main.c} that was submitted along with this report.

\end{document}